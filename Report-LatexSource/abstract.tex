\begin{abstract}
    The goal of this project was to explore the usage and techniques of different data mining techniques, and compare their advantages and disadvantages by using a data set sourced from real-world studies. The data set itself was created through the school results of 649 students from a Portuguese secondary school, and a questionnaire that was provided to these students. The grades and absences of the students were extracted from each student's school reports, while the questionnaire was used to gather demographic, social, and school related features. In this report, the students were classified based on their average grade across the three terms reported. In particular, binary classification was used to determine whether the student's grade was ``Above'' or ``Below'' average, and 5-level-classification was used to assign one of five labels from ``Very Good'' to ``Fail'' to the average grade. In order to classify the students, four different data mining techniques were applied: k-Nearest Neighbour, Decision Trees, Naïve Bayes, and Neural Networks. In addition to classifying the average grade, this project also aimed to predict the average grade of a student through the use of linear regression by way of feature selection and calculation and through neural networks. These methods were trained and tested by using separate training and test sets calved from the full data set. The records for the test set were chosen randomly and consist of one-third of all students from the full data set. The algorithms chosen were either from scratch or used various machine-learning libraries such as scikit-learn \cite{sklearn} and Google's TensorFlow \cite{tf}.The results of these custom-made implementations were compared to the results from the popular Java library Weka \cite{weka}. Finally, the calculations from all methods were analyzed, and it was determined if accurate classifications and regressions can be made from this data set. 
\end{abstract} 