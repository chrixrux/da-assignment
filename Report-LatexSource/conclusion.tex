\section{Conclusion}
This experiment attempted to use academic and demographic records from students taking two subjects in two schools in Portugal to classify and extrapolate the average grade of a student. This data set was first preprocessed based on sound principles reducing the number of outliers in the set, then subsequently mined using custom and provided implementations of the linear regression, naïve Bayes classification, k-Nearest Neighbor, artificial neural networks, and decision trees algorithms. However, due to the quality of the data set itself, accurate classifiers were not able to be constructed, only at most giving an accuracy of 75\% in the best case. As a result of the inability to generate relatively accurate classifiers, it is not yet possible to begin to envision effective manners of improving student education based on their demographic information.